\documentclass[DM,toc]{lsstdoc}
% lsstdoc documentation: https://lsst-texmf.lsst.io/lsstdoc.html

% Generated by Makefile
\input{meta}

% Package imports go here.

% Local commands go here.

% If you want glossaries, uncomment:
% \input{aglossary.tex}
% \makeglossaries

\title{Characterization Metric Report: Science Pipelines Version 27.0.0}
% \setDocSubtitle{Optional subtitle}

\author{%
Jeff Carlin
}

\setDocRef{DMTR-431}
\setDocUpstreamLocation{\url{https://github.com/lsst-dm/DMTR-431}}
\date{\vcsDate}
% \setDocCurator{The Curator of this Document}

\setDocAbstract{%
Add abstract text.
}

% Revision history.
% Order: oldest first.
% Fields: VERSION, DATE, DESCRIPTION, OWNER NAME.
% See LPM-51 for version number policy.
\setDocChangeRecord{%
  \addtohist{1}{2024-06-05}{Unreleased.}{Jeff Carlin}
}

\begin{document}

\maketitle

% ADD CONTENT HERE
% You can also use the \input command to include several content files.

\appendix

% Include all the relevant bib files.
% https://lsst-texmf.lsst.io/lsstdoc.html#bibliographies
\section{References} \label{sec:bib}
\renewcommand{\refname}{} % Suppress default Bibliography section
\bibliography{local,lsst,lsst-dm,refs_ads,refs,books}

% Make sure lsst-texmf/bin/generateAcronyms.py is in your path
\section{Acronyms} \label{sec:acronyms}
\addtocounter{table}{-1}
\begin{longtable}{p{0.145\textwidth}p{0.8\textwidth}}\hline
\textbf{Acronym} & \textbf{Description}  \\\hline

CCD & Charge-Coupled Device \\\hline
DM & Data Management \\\hline
DMSR & DM System Requirements; LSE-61 \\\hline
DMTN & DM Technical Note \\\hline
DMTR & DM Test Report \\\hline
DRP & Data Release Production \\\hline
FAQ & Frequently Asked Question \\\hline
HSC & Hyper Suprime-Cam \\\hline
HSC-SSP & HSC Subaru Strategic Program \\\hline
KPM & Key Performance Metric \\\hline
LPM & LSST Project Management (Document Handle) \\\hline
LSE & LSST Systems Engineering (Document Handle) \\\hline
LSR & LSST System Requirements; LSE-29 \\\hline
LSST & Legacy Survey of Space and Time (formerly Large Synoptic Survey Telescope) \\\hline
PSF & Point Spread Function \\\hline
Pan-STARRS & Panoramic Survey Telescope and Rapid Response System \\\hline
SRD & LSST Science Requirements; LPM-17 \\\hline
\end{longtable}

% If you want glossary uncomment below and comment out the two lines above.
% \printglossaries

\end{document}
